\section{Background information}
We are in the 21st century and space travel is all the more a reality now than it was a century prior. There is an expanded requirement for knowledge on how movement outside our earth can be ventured into using safe routes with an exactness that is of the micrometer-scale. To make this possible, researchers have appropriately gathered data of heavenly bodies motions and extraterrestrial environments. With these equations that describe with accuracy the movement of the bodies were derived. Using this information mankind has been able to send vehicles into space. This has prompted space exploration and all the more significantly space correspondence. Space communication involves the satellites that have been sent into space in the endeavor to narrow the communication gap on the earth below. This has brought about application GPS and Satellite correspondence (which includes the exchange of information through satellites and ground stations). In this task we expect to explore on satellite - earth material science by determining conditions that will suitably portray this framework and by making a straightforward model of the same. We will reach determinations on whether Kenya as nation should wander into space by completing a cost assessment and different variables. 

\section{Statement of the problem}
Kenya, like many other countries in Africa,  has not been able to send a satellite into space despite being one of the leading nations in Africa. Kenya is a huge consumer of satellite technology owing to the number of corporations in this country who depend on it to run their activities. But this satellite technology is outsourced. What does Kenya lack that it is not able to send satellites into space?
However, it is only until recently that Kenya was able to send its first pico-satellite into space in a joint venture with Japan. This was done in collaboration with the University of Nairobi. This heralded a new dawn with Kenya having an interest in space.

We are going to look into this matter from an academic angle inquiring the physics involved in the operation of sending a satellite in space and keeping it in space. Due to the scope of this research and the time given, we will focus mainly on the physics of keeping a satellite in orbit. What are the equations involved in keeping a satellite in motion around the planet earth? What is the cost, in monetary terms, involved in keeping a satellite in orbit? Is Kenya able to meet this specifications?

\section{Justification and Significance of the Study}
Kenya is on the verge of constituting a space agency of which the administration of sending satellites into space on behalf of Kenya will be its prerogative. After many years of absence in the space scene will Kenya be able to handle these leap onto new grounds? In this research we will look into answering these questions. 
We will develop a system of equations that will be used to develop simulations that will show that Kenya is better prepared to launch into space. This study will also show what Kenya might expect to incur in terms of the monetary costs on achieving the objective of sending a satellite into space hence avoiding the factor of surprise when Kenya begins to dig deep into the pocket.
\section{Objectives}
\subsection{Main Objective}
To perform a computer model simulation of an isolated Earth-Satellite system based on the two-body problem astrodynamics and compare simulated data with past collected to prove the correctness of the model.
\subsection{Specific Objectives}
1. Review the physics of the two-body problem.\\                                                                                                                                                                       2. Build a computer model for a Earth satellite system in 3-dimensions.\\
3. Apply the model for the Kenyan satellite 1KUNS-PF.
\section{Methodology}
This research will involve deriving equations simplifying the two-body problem which will later be encapsulated in one equation.  This equation will be changed into a computer program that will simulate the interaction between the earth and the satellites that move around it. This simulation will be done in 3 dimensions using Python as it is rich in scientific libraries that will enable the simulation. We will then proceed to determine the accuracy of the model.